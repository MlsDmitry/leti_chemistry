\large
\textbf{Цель работы}

Ознакомление с методами получения кислот, солей, оснований и с их
химическими свойствами.

\begin{center}
    \large
    \textbf{Основные теоретические положения}
\end{center}
% \chapter{Основные теоретические положения}%
% \label{cha:Основные теоретические положения}

\section{Оксиды}%
\label{sec:Оксиды}

\subsection{Определение и классификация}
Оксиды - химические соединения, состоящие из двух элементов, один из
которых кислород в степени окисления «-2». Оксидами называются соединения
элементов с кислородом, в которых атомы кислорода химически не связаны друг с
другом.

Оксиды делятся на солеобразующие и несолеобразующие. Последних довольно
мало (оксид азота (I) - $\ce{N2O}$, оксид азота (II) - $\ce{NO}$, оксид углерода (II) - $\ce{CO}$, оксид
кремния (II) - $\ce{SiO}$), они не образуют солей ни с кислотами, ни со щелочами.
Солеобразующие оксиды делятся на основные, кислотные и амфотерные.
\\

\subsection{Основные оксиды}

Основными называются оксиды ($\ce{Na2O}$, $\ce{MgO}$, $\ce{CuO}$ и др.), которым
соответствуют гидроксиды, относящиеся к классу оснований. Основные оксиды
образуют металлы, проявляющие в соединениях валентность I, II, III (не выше как
правило).

\subsubsection{Химические свойства основных оксидов}

\begin{enumerate}
    \item Основный оксид + вода = основание
        $$\ce{Na2O + H2O -> 2NaOH}$$

    \item Основный оксид + кислота = соль + вода
        $$\ce{CaO + 2HCl -> CaCl2 + H2O}$$

    \item Основный оксид + кислотный оксид = соль
        $$\ce{Li2O + CO2 -> Li2CO3}$$
\end{enumerate}

\subsection{Кислотные оксиды}
Кислотными называют оксиды ($\ce{SO2}$, $\ce{CO2}$ и др.), которым соответствуют
гидроксиды, относящиеся к классу кислот. Реагируя с основаниями, эти оксиды
образуют соль и воду. Кислотные оксиды - это, главным образом, оксиды неметаллов
2с ковалентной связью. Степень окисления металлов в кислотных оксидах, как
правило, больше +4 ($\ce{V2O5}$, $\ce{CrO3}$ , $\ce{Mn3.2}$ ).

\subsubsection{Химические свойства кислотных оксидов}

\begin{enumerate}
    \item Кислотный оксид + вода = кислота
        $$\ce{P2O5 + 3H2O -> 2H3PO4}$$

    \item Кислотный оксид + основание = соль + вода
        $$\ce{SO3 + 2NaOH -> Na2SO4 + H2O}$$

    \item Кислотный оксид + основный оксид = соль
        $$\ce{CaO + SO3 -> CaSO4}$$
\end{enumerate}

\subsection{Амфотерные оксиды}%
\label{sec:}

Амфотерными называются оксиды ($\ce{BeO}$, $\ce{ZnO}$, $\ce{PbO}$, $\ce{SnO}$, а также оксиды
металлов со степенью окисления III и IV, например, $\ce{Al2O3}$, $\ce{Cr2O3}$ и др.), которые
обладают двойственными свойствами и ведут себя в одних условиях как основные, а
в других - как кислотные, т. е. образуют соли при взаимодействии как с кислотами,
так и с основаниями.

\subsubsection{Химические свойства амфотерных оксидов}

\begin{enumerate}
    \item Амфотерный оксид + кислота = соль(1 типа) + вода
        $$\ce{ZnO + H2SO4 -> ZnSO4 + H2O}$$

    \item Амфотерный оксид + щёлочь = соль(2 типа) + вода
        $$\ce{ZnO + 2NaOH -> Na2ZnO2 + H2O}$$
\end{enumerate}

Многие элементы проявляют переменную степень окисления, образуют оксиды
различного состава, что учитывается при названии оксида указанием валентности
элемента: $\ce{CrO}$ - оксид хрома (II), $\ce{Cr2O3}$ - оксид хрома (III), $\ce{CrO3}$ - оксид хрома (VI).

\subsubsection{Получение амфотерных оксидов}

\begin{enumerate}
    \item Взаимодействие металлов с кислородом:
        $$\ce{4Li + O2 -> 2Li2O}$$

    \item Взаимодействие неметаллов с кислородом:
        $$\ce{S + O2 -> SO2}$$

    \item Разложение оснований при нагревании:
        $$\ce{Cu(OH)2 -> CuO + H2O}$$

    \item Разложение некоторых солей при нагревании:
        $$\ce{CaCO3 -> CaO + CO2}$$
\end{enumerate}

\section{Основания}

\subsection{Определение и классификация}
Основания - это гидроксиды металлов, при диссоциации которых
образуются гидроксид-ионы ($\ce{OH-}$) и основные остатки:
$$\ce{Cu(OH)2 <-> Cu(OH)+ + OH-}$$

Кислотность оснований определяется числом гидроксид-ионов в молекуле
основания. Многокислотные основания диссоциируют ступенчато:
$$\ce{Cu(OH)+ <-> Cu2+ + OH-}$$

Названия оснований составляют из слова «гидроксид» и названия металла
($\ce{NaOH}$ - гидроксид натрия) с указанием валентности, если металл образует несколько
оснований, например: $\ce{Fe(OH)2}$ - гидроксид железа (II), $\ce{Fe(OH)3}$ - гидроксид железа (III).

По растворимости в воде различают: основания, растворимые в воде - щелочи (гидроксиды щелочных и щелочно-земельных металлов) и основания, нерастворимые в воде, например $\ce{Cu(OH)2}$ , $\ce{Fe(OH)3}$, $\ce{Cr(OH)3}$ и др.

\subsection{Химические свойства оснований}

\begin{enumerate}
    \item Водные растворы щелочей изменяют окраску индикаторов: в их присутствии
        лакмус синеет, бесцветный фенолфталеин становится малиновым, метиловый
        оранжевый - желтым.
    \item Кристаллы щелочей при растворении в воде полностью диссоциируют, то есть
        распадаются на положительно заряженные ионы металла и отрицательно
        заряженные гидроксид-ионы.
        $$\ce{4NaOH -> NA+ + OH-}$$
    \item Основания реагируют с кислотными оксидами и кислотами с образованием соли
        и воды и не реагируют с основными оксидами и щелочами.
        $$\ce{Ca(OH)2 + CO2 -> CaCO3 v + H2O}$$
        $$\ce{NaOH + HCl -> NaCl + H2O}$$
    \item Нерастворимые основания разлагаются при нагревании.
        $$\ce{Cu(OH)2 -> CuO + H2O}$$
    \item Нерастворимые основания взаимодействуют с кислотами, образуя соль и воду.
        $$\ce{Cu(OH)2 + H2SO4 -> CuSO4 + 2H2O}$$
    \item Некоторые нерастворимые основания могут взаимодействовать с некоторыми
        кислотными оксидами, образуя соль и воду.
        $$\ce{Cu(OH)2 + SO3 -> CuSO4 + H2O}$$
    \item Щёлочи могут взаимодействовать с растворимыми в воде солями.
        $$\ce{2NaOH + CuSO4 -> Na2SO4 + Cu(OH)2 v}$$
    \item Малорастворимые щёлочи при нагревании разлагаются на оксид металла и воду.
        $$\ce{Ca(OH)2 -> CaO + H2O ^}$$
\end{enumerate}

Амфотерные гидроксиды проявляют как основные, так и кислотные свойства.
К ним относятся, например, $\ce{Al(OH)3}$ , $\ce{Zn(OH)2}$ , $\ce{Cr(OH)3}$ , $\ce{Be(OH)2}$ и др.

\subsection{Получение}

\begin{enumerate}
    \item Щелочи получают, растворяя в воде оксиды щелочных и щелочноземельных
        металлов.
        $$\ce{2Na + 2H2O -> 2NaOH + H2 v }$$
    \item Щёлочи образуются при взаимодействии оксидов щелочных и щелочноземельных
        металлов с водой. При этом протекает реакция соединения.
        $$\ce{Li2O + H2O -> 2LiOH}$$
    \item В промышленности гидроксид натрия и калия получают путём электролиза:
        пропускают постоянный электрический ток через раствор хлорида натрия или
        калия.
        $$\ce{2NaCl + 2H2O- -> 2NaOH + H2 ^  + Cl2 ^} $$
    \item Чтобы получить нерастворимое основание, следует к раствору соли
        соответствующего металла добавить раствор щёлочи.
        $$\ce{CuCl2 + 2KOH -> Cu(OH)2 v + 2KCl}$$
\end{enumerate}

\section{Кислоты}
\subsection{Определение и классификация}
Кислоты - это электролиты, при диссоциации которых в качестве катионов
образуются ионы водорода ($\ce{H+}$ ) и анионы кислотных остатков.
По наличию кислорода в своем составе кислоты делятся на бескислородные
(например, $\ce{HCl}$, $\ce{HBr}$, $\ce{H2S}$) и кислородосодержащие (например, $\ce{HNO3}$ , $\ce{H2SO4}$ , $\ce{H3PO4}$ )

\subsection{Химические свойства кислот}

\begin{enumerate}
    \item В растворах кислот индикаторы меняют свою окраску: лакмус и метилоранж
        становятся красными.
    \item Кислоты взаимодействуют с металлами, стоящими левее водорода в
        электрохимическом ряду напряжений (ряд активностей металлов), образуют
        соли и выделяют водород. Водород не выделяется при взаимодействии металлов
        с азотной и концентрированной серной кислотами.
        $$\ce{Mg + 2HCl -> MgCl2 + H2 ^}$$
    \item Кислоты реагируют с основными и амфотерными оксидами, образуя соль и
        воду.
        $$\ce{K2O + 2HNO3 -> 2KNO3 + H2O}$$
        $$\ce{Al2O3 + 6HCl -> 2AlCl3 + 3H2O}$$
    \item Взаимодействуют с основаниями и с амфотерными гидроксидами.
        $$\ce{KOH + HNO3 -> KNO3 + H2O}$$
    \item Кислоты реагируют с растворами солей, если в результате реакции один из
        продуктов выпадает в осадок.
        $$\ce{H2SO4 + BaCl2 -> BaSO4 v + 2HCl}$$
    \item Разложение кислородсодержащих кислот.
        $$\ce{H2CO3 <-> H2O + CO2 v}$$
    \item Если кислота, которая вступает в реакцию, является сильным электролитом, то
        кислота, которая образуется — слабым.
        $$\ce{2HCl + CaCO3 -> CaCl2 + H2O + CO2 ^}$$
\end{enumerate}

\subsection{Получение кислот}

\begin{enumerate}
    \item   Из неметаллов и водорода с последующим растворением образовавшегося газа в воде:
        $$\ce{Cl2 + H2 -> 2HCl}$$
    \item При действии сильных кислот на соли более слабых или летучих бескислородных кислот:

        $$\ce{2HCl + Na2S -> 2NaCl + H2S}$$
        $$\ce{2H+ + S2- -> H2S}$$

        Получение кислородсодержащих кислот

    \item Взаимодействием кислотного оксида и воды. Оксид кремния(IV) SiO2 с водой не реагирует!

        $$\ce{SO2 + H2O -> H2SO3}$$
    \item При действии сильных кислот на соли более слабых или летучих кислородсодержащих кислот:

        $$\ce{2HCl + Na2CO3 -> 2NaCl + H2CO3}$$
        $$\ce{2H+ + CO3^2- -> H2CO3}$$
\end{enumerate}

\section{Соли}

\subsection{Определение и классификация}
Соли - электролиты, при диссоциации которых образуются катионы основных остатков и анионы кислотных остатков.


По составу соли подразделяют на средние (нормальные), кислые (гидросоли), основные (гидроксосоли), двойные, смешанные и комплексные.


\begin{tabular}{| p{4.1cm} | p{4.1cm} | p{4.1cm} }
    \hline
    Средние (нормальные) - продукт полного замещения атомов водорода в кислоте на металл $\ce{AlCl3}$ & Кислые(гидросоли) - продукт неполного замещения атомов водорода в кислоте на металл $\ce{KHSO4}$ & Основные (гидроксосоли) -продукт неполного замещения ОН-групп основания на кислотный остаток $\ce{FeOHCl}$ \\ \hline
    Двойные - содержат два разных металла и один кислотный остаток $\ce{KNaSO4}$ & Комплексные $\ce{[Cu(NH3)4]SO4}$ & Смешанные - содержат один металл и несколько кислотных остатков $\ce{CaClBr}$
\end{tabular}

\subsection{Химические свойства солей}

\begin{enumerate}
    \item  Диссоциация. Средние, двойные и смешанные соли диссоциируют одноступенчато. У кислых и основных солей диссоциация происходит ступенчато.

        $$\ce{NaCl  Na+ + Cl-}$$
        $$\ce{KNaSO4 <-> K+ + Na+ + SO4^2-}$$
        $$\ce{CaClBr <-> Ca2+ + Cl -+ Br-}$$
    \item Взаимодействие с индикаторами. В результате гидролиза в растворах солей накапливаются ионы Н+ (кислая среда) или ионы ОН- (щелочная среда). Гидролизу подвергаются растворимые соли, образованные хотя бы одним слабым электролитом. Растворы таких солей взаимодействуют с индикаторами:

        индикатор + $\ce{H+ (OH-) ->}$  окрашенное соединение.

    \item Разложение при нагревании. При нагревании некоторых солей они разлагаются на оксид металла и кислотный оксид:
        % $$\ce{CaCO3-> CaO + CO2­}$$
        соли бескислородных кислот при нагревании могут распадаться на простые вещества:
        % $$\ce{2AgCl -> Ag + Cl2­}$$
        Соли, образованные кислотами-окислителями, разлагаются сложнее:
        % $$\ce{2KNO3-> 2KNO2 + O2­}$$
    \item Взаимодействие с кислотами: Реакция происходит, если соль образована более слабой или летучей кислотой, или если образуется осадок.
        % $$\ce{2HCl + Na2CO3 -> 2NaCl + CO2­ + H2O}$$
        $$\ce{2hcl + na2co3 -> 2nacl + co2- + h2o}$$
        % $$\ce{СaCl2 + H2SO4 -> CaSO4¯ + 2HCl}$$
        $$\ce{CaCl2 + h2so4 -> caso4- + 2hcl}$$
        Основные соли при действии кислот переходят в средние:
        $$\ce{FeOHCl + HCl -> FeCl2 + H2O}$$
        Средние соли, образованные многоосновными кислотами, при взаимодействии с ними образуют кислые соли:
        $$\ce{Na2SO4 + H2SO4 -> 2NaHSO4}$$
    \item Взаимодействие со щелочами. Со щелочами реагируют соли, катионам которых соответствуют нерастворимые основания.
        $$\ce{CuSO4 + 2NaOH -> Cu(OH)^2- + Na2SO4}$$
    \item Взаимодействие друг с другом. Реакция происходит, если взаимодействуют растворимые соли и при этом образуется осадок.
        $$\ce{AgNO3 + NaCl -> AgCl- + NaNO3}$$
    \item Взаимодействие с металлами. Каждый предыдущий металл в ряду напряжений вытесняет последующий за ним из раствора его соли:
        $$\ce{Fe + CuSO4 -> Cu- + FeSO4}$$
        $$\ce{Fe + Cu2+ -> Cu- + Fe2+}$$

        \small
        Li, Rb, K, Ba, Sr, Ca, Na, Mg, Al, Mn, Zn, Cr, Fe, Cd, Co, Ni, Sn, Pb, H, Sb, Bi, Cu, Hg, Ag, Pd, Pt, Au
    \item Электролиз (разложение под действием постоянного электрического тока). Соли подвергаются электролизу в растворах и расплавах:
        % $$\ce{2NaCl + 2H2O H2­ + 2NaOH + Cl2­}$$
        $$\ce{2NaCl + 2h2o -> h2- + 2naoh + cl2-}$$
    \item Взаимодействие с кислотными оксидами.
        % $$\ce{CO2 + Na2SiO3  -> Na2CO3  + SiO2}$$
        $$\ce{co2 + na2sio3 -> na2co3 + sio2}$$

\end{enumerate}

\subsection{Получение солей}

\begin{enumerate}
    \item   1) Взаимодействием металлов с неметаллами:
        2Na + Cl2 ® 2NaCl.
    \item Взаимодействием основных и амфотерных оксидов с кислотными оксидами:
        $$\ce{CaO + SiO2 CaSiO3}$$
        $$\ce{ZnO + SO3 ZnSO4}$$
    \item Взаимодействием основных оксидов с амфотерными оксидами:
        $$\ce{Na2O + ZnO -> Na2ZnO2}$$
    \item Взаимодействием металлов с кислотами:
        $$\ce{2hcl + Fe -> fecl2 + h2-}$$
    \item Взаимодействием основных и амфотерных оксидов с кислотами:
        $$\ce{Na2O + 2HNO3 -> 2NaNO3 + H2O}$$
        $$\ce{ZnO + H2SO4 -> ZnSO4 + H2O}$$
    \item Взаимодействием амфотерных оксидов и гидроксидов со щелочами:
        В растворе:
        $$\ce{2naoh + zno + h2o -> na2[zn(oh)4]}$$
        $$\ce{2oh- + zno + h2o -> [zn(oh)4]2-}$$

        При сплавлении с амфотерным оксидом:
        $$\ce{2NaOH + ZnO -> Na2ZnO2 + H2O}$$
        В растворе:
        $$\ce{2NaOH + Zn(OH)2 -> Na2[Zn(OH)4]}$$
        $$\ce{2OH-  +  Zn(OH)2 -> [Zn(OH)4]2-}$$
        При сплавлении:
        $$\ce{2NaOH + Zn(OH)2 -> Na2ZnO2 + 2H2O}$$
    \item Взаимодействием гидроксидов металлов с кислотами:
        $$\ce{Ca(OH)2 + H2SO4 -> CaSO4- + 2H2O}$$
        $$\ce{Zn(OH)2 + H2SO4 -> ZnSO4 + 2H2O}$$
    \item Взаимодействием кислот с солями:
        $$\ce{2hcl + na2s -> 2nacl + h2s-}$$
    \item Взаимодействием солей со щелочами:
        $$\ce{znso4 + 2naoh -> na2so4 + zn(oh)2-}$$
    \item Взаимодействием солей друг с другом:
        $$\ce{AgNO3 + KCl -> AgCl- + KNO3}$$
\end{enumerate}

\clearpage
